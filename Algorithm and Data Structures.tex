% Template to be used while publishing    
% a scientific work (PhD Thesis etc.)
% at the EDOC-Server of HU-Berlin                      
%    developed 2003-2008 by
% AG Elektronisches Publizieren, 
% Computer- und Medienservice,
% Humboldt-Universitaet zu Berlin
%    with friendly support of
% TeX-Stammtisch in Berlin     

% $Revision: 19 $
% $HeadURL: svn+ssh://ryckojox@svn.cms.hu-berlin.de/svn/projects/epub/latex/hudiss/mustermann.tex $
% $Date: 2009-07-03 15:49:19 +0200 (ptk, 03 lip 2009) $
% $Author$
% $Id: mustermann.tex 19 2009-07-03 13:49:19Z ryckojox $
                             
% Questions, comments, support:
%    edoc-latex@cms.hu-berlin.de    

% Documentation and information about the conditions of a publication:
%    http://edoc.hu-berlin.de/e_autoren/latex          

% Upload:
%    https://edoc.hu-berlin.de/cgi/dokupload/dokupload.cgi

\documentclass[openright,twoside,headsepline,bibtotoc]{scrbook}[2007/12/24]

% The following package is necessary.
% To change the default options of the packages, use the key=value interface.
% See the documentation for details.
% If you need to use any special characters or LaTeX-commands 
% within the options, use \hudisssetup{} after loading this package,
% otherwise they will NOT work correctly.
\usepackage[
    inputenc=utf8, % default - latin1
    fontset=palatino, % other possible parameters: lmodern, times, palatino
    natbib={round,authoryear}, %include to use the natbib-package % sort sortiert nach Name
    % jurabib={}, %include to use the jurabib-package
    % apacite={}, %include to use the apacite-package
    hints=false, % set to 'false' for submission
    checktabu=false, % set to 'false' for submission
    draft=false, % set to 'false' for submission
    qserie=false
 ]{hudiss}

%\usepackage{layouts} % nur f�r Ausgabe der Textbreite
% Breite hier: B=14,6979 cm. Ergibt H=8,3988 cm f�r 7/4 Verh�ltnis
% f�r kleine Bilder nehmen wir 6.5/4: BxH=8.819x5.427

                
\hudisssetup{%
    titlepagefont={\Large\sffamily} % Use to change the titlepage font
}

% Fill out all the metadata here:
\hudissmetadata{%
    authorprefix={}, % e.g. Dipl.-Inf.
    authorfirstname={Manuel}, % first name
    authorsurname={Lehmann}, % surname
    authorsuffix={}, % e.g. Ph.D.
    %authoradd={geboren am 23.\,10.~1980 in Siegen}, % date and place of birth
    doctitle={Algorithms and Data Structures}, % title of the thesis
    %docsubtitle={for one euro}, % subtitle of the thesis
    docsubject={Script}, % subject of the thesis (used in the properties of the pdf-document)
    %approvala={Prof. Dr. I. M. Sokolov, Humboldt-Universit�t zu Berlin}, % approvals: a-e
    %approvalb={Prof. Dr. Dr. J. Kurths, Potsdam-Institut f�r Klimafolgenforschung},
    %approvalc={Prof. Dr. B. Blasius, Carl von Ossietzky Universit�t, Oldenburg},
    % approvald={},
    % approvale={},
    %degree={doctor rerum naturalium\\(Dr. rer. nat.)}, % e.g. Dr. Rer. Nat.
    subject={Informatik}, % e.g. Informatik
    faculty={Mathematisch-Naturwissenschaftlichen Fakultät I}, % in Dativ/Genitiv! e.g. Mathematisch-Wissenschaftlichen Fakult\"at II
    university={Humboldt-Universität zu Berlin}, % e.g. Humboldt-Universit\"at zu Berlin
    %dean={Prof. Stefan Hecht, PhD}, % dean of the faculty
    president={Prof. Dr. Jan-Hendrik Olbertz}, % president of the university
    %datesubmitted={11. November 2013}, % the date of the submission
    %dateexam={06. November 2013}, % the date of your last exam
    %keywordsen={Complex Network, Epidemiology, Temporal Network, Statistical Physics}, % english keywords comma separated
    %keywordsde={Komplexes Netzwerk, Epidemiologie, zeitabh�ngiges Netzwerk, statistische Physik} % german keywords comma separated
}

% If you wish to load any further packages, 
% make any own adjustments (e. g. for the package fancyhdr)
% or define any own commands
% put ALL of them in the following file:
\KOMAoptions{numbers=noenddot}
\usepackage{amsmath,amssymb,amsfonts,amsthm,epigraph,scrpage2}
\usepackage[ngerman,english]{babel}
\usepackage{centernot} % fuer Durchstreichung
\usepackage{multirow}
\usepackage{pdfpages}


\usepackage{currvita}
\renewcommand*{\cvheadingfont}{\large\bfseries} % CV Titel
\renewcommand*{\cvlistheadingfont}{\bfseries\sffamily} % sub-�berschriften
\renewcommand*{\cvlabelfont}{\sffamily} % items

\definecolor{Cayenne}{rgb}{0.502,0.0,0.0}
\definecolor{Steel}{rgb}{0.4,0.4,0.4}
\definecolor{Tri_blue}{rgb}{0.039,0.5098,0.8}
\definecolor{Tri_yellow}{rgb}{0.5529,0.5451,0.0549}


%\setcounter{secnumdepth}{3} % sub subsections numbering
%\setcounter{tocdepth}{3} % subsubsections inTOC

\usepackage[format=plain,singlelinecheck=false, font={sf,small},labelfont={bf,color=Steel}]{caption}
\DeclareCaptionLabelSeparator{cayenne_period}{\textcolor{Cayenne}{.} }
\captionsetup{labelsep=cayenne_period}

% Colors
\addtokomafont{chapter}{\color{Steel}}
\addtokomafont{section}{\color{Steel}}
\addtokomafont{subsection}{\color{Steel}}
\addtokomafont{subsubsection}{\color{Steel}}
\addtokomafont{paragraph}{\color{Steel}}
\addtokomafont{disposition}{\color{Steel}}
\addtokomafont{pagehead}{\color{Steel}}
\renewcommand{\pnumfont}{\color{Steel}} 
\addtokomafont{headsepline}{\color{Steel}} 
\pagestyle{scrheadings}

% Textcolor for chapters in TOC
\makeatletter
\let\stdl@chapter\l@chapter
\renewcommand*{\l@chapter}[2]{%
  \stdl@chapter{\textcolor{black}{#1}}{\textcolor{black}{#2}}}
\makeatother

%  labels in description environments
\renewcommand{\descriptionlabel}{\hspace\labelsep{}\sffamily\small\bfseries{}\color{Steel}{}}

%\makeatletter % dot after sections and all below
%\let\std@sect\@sect
%\def\@sect#1#2#3#4#5#6[#7]#8{\std@sect{#1}{#2}{#3}{#4}{#5}{#6}[#7.]{#8\color{Cayenne}{.}}}
%\makeatother
\usepackage{acronym}

\usepackage[leftcaption]{sidecap} % inner, outer,left,right
\sidecaptionvpos{figure}{t}

% Papiergr��e
%\setlength{\paperwidth}{21cm}
%\setlength{\paperheight}{25cm}
%\recalctypearea
%\usepackage[pass]{geometry}
%\usepackage[cross,a4,center]{crop}

%% Flattersatz
%\usepackage[document]{ragged2e} % Flattersatz
%\setlength{\RaggedRightParindent}{1em} % evtl. parskip


%% Sans Serif
%\usepackage{cmbright}
%\renewcommand{\familydefault}{\sfdefault}
%% Palatino
%\usepackage[sc]{mathpazo}
%\linespread{1.05}         % Palatino needs more leading (space between lines)
%\setkomafont{sectioning}{\normalcolor\bfseries} % Kapitel�berschriften

%%% Kapitel�berschriften: Mit gro�en Zahlen
%\usepackage{titlesec}
%\titleformat{\chapter}[display]
%{\bfseries\Large}
%{ %\Huge\textsc{\chaptertitlename} % f�r das Wort 'Kapitel'
%\hfill\fontsize{120}{70}\selectfont\color{lightgray}\textbf{\thechapter}}
%{-2ex}
%%{\filleft\fontsize{50}{70}\selectfont\scshape} % Kapit�lchen oder...
%{\filleft\fontsize{50}{70}\selectfont\textbf} % ...oder keine Kapit�lchen
%[\vspace{0ex}]
%
%%%% Part�berschriften
%\titleformat{\part}[display]
%{\bfseries\Large}
%{ %\Huge\textsc{\chaptertitlename} % f�r das Wort 'Kapitel'
%\hfill\fontsize{120}{70}\selectfont\color{lightgray}\textbf{\thepart}}
%{-2ex}
%{\filleft\fontsize{50}{70}\selectfont\scshape} % Kapit�lchen oder...
%%{\filleft\fontsize{50}{70}\selectfont\textbf} % ...oder keine Kapit�lchen
%[\vspace{0ex}]


\newcommand{\ER}{Erd\H{o}s-R\'enyi }
\newcommand{\BA}{Barab\'asi-Albert }
\newcommand{\mean}[1]{\left< #1 \right>}
\newcommand{\abs}[1]{\left| #1 \right|}
\newcommand{\norm}[1]{\lVert#1\rVert}
\newcommand{\mat}[1]{\mathbf{#1}}
\newcommand{\tgraph}{\mathcal{G}}

\theoremstyle{definition} % non-italic
\newtheorem{annahme}{Annahme} % braucht amsthm
\newtheorem{definition}{Definition}
\newtheorem{theorem}{Theorem}
\newtheorem{satz}{Satz}
\newtheorem{frage}{Frage}
%\input{watermarks/watermark.tex}
\DeclareMathOperator{\nnz}{nnz}

% + Graphicspath nach begin document

% aus Doi hyperref machen
%\newcommand*{\doi}[1]{\href{http://dx.doi.org/\detokenize{#1}}{doi: \detokenize{#1}}}




% The order of the parts in the document is only our suggestion,
% you can change it, if you wish.
% Don't put any other text between those commands.
% Do not remove the \*matter macros.
% Use standard macros to include new chapters.



%%%%%    \includeonly{chapters/Part1/01-Introtext}
% DIV is 10, BCOR is 0mm
\begin{document}
\graphicspath{{./images/}{./images_gnu_tex}}

\selectlanguage{english}
    \frontmatter
    	%\includepdf{Thesis_Cover.pdf}
	%\cleardoublepage
        \maketitle
        \cleardoublepage
        %\cleardoublepage

\null\vfill\itshape

\begin{flushright}
	Ich widme diese Arbeit \\
	meiner Familie und meinen Freunden
\end{flushright}
\thispagestyle{empty}
\upshape\cleardoublepage


        \selectlanguage{english}
\begin{abstract}
LOREM IPSE

\paragraph{Keywords\color{Cayenne}{:}} Complex Network, Epidemiology, Temporal Network, Statistical Physics
\end{abstract}

\cleardoublepage
        \tableofcontents    
        %\chapter*{}
\section*{List of abbreviations}
%\addcontentsline{toc}{chapter}{List of abbreviations}
%\thispagestyle{plain}

\begin{acronym}[nnzzzzzzz] %5 l�ngste Abk�rzung ineckigen Klammern zur Ausrichtung
%\setlength{\itemsep}{-\parsep}

\acro{}[\color{Steel}{Static networks}\color{Cayenne}{.}]{}
\acro{G}[$G$]{Network/Graph. A tuple $G=(V,E)$ of a set of nodes $V$ and a set of edges~$E$.}

\acro{}[]{}
\acro{}[\color{Steel}{Epidemic models}\color{Cayenne}{.}]{}
\acro{alpha}[$\alpha $]{Infection rate.}

\end{acronym}

    \mainmatter
        % Part 1
        \chapter{Introdcution}
\section{•}
%        - introduction
%        		- first approach
%        		- what is an algorithm
%        		- history
%        		- algorithm of euklid
%        		
%        	_CHAPTER ONE_ definitions 
		\include{chapters/chapter_01/algorithm.tex}
%        - Algorithm
%        		- definition
%        		- correct algorithm
%        		- proof of correctness
%        		- properties of algorithms
%        		- runtimes
%        		- complexity (times/space)
		\include{chapters/chapter_01/data_structure.tex}
%        	- Data structures
%        		- what?
%        		- selection sort -> array/ linked list
%        		- why?
%        	
%        	_CHAPTER TWO_ complexity
%        	- Efficiency of algorithms
%        		- reference mashine / computational complexity
%        		- reference mashine
%        		- cost model
%        		- example x^y -> cost model vs count ops in RAM
%        		- conclusion?
%        	- O-Notation
%        		- definition
%        		- explanation
%        		- calculating with complexities
%        		- O-calculus
%        		- Gamma-Notation
%        		- further notations
%        	
%        	_CHAPTER THREE_ data types
%        	- abstract data types ADT
%			- why?
%			- classic box problem and voronoi diagrams
%        		- example
%        		- example: list vs points?
%        		- list/stacks/queues
%        		- modeling more details
%        		- reusing existing adts
%        		- axioms
%        		- remarks about programm languages and java in particular
        		
        	\include{chapters/chapter_04/max_subarray.tex}
%       	_CHAPTER FOUR_ max subarray
%        	- the problem
%    		- types of approches /algorithms
%   	    - example: (explanation+pseudocode+complexity)
%       		- greedy solution
%        		- exhausative solution
%        		- divide and conquer
%        		- different approch




        			
%        	_CHAPTER FIVE_ lists
%        	- ADT lists
%        	- using array
%        		- insert, delete, search
%        		- array of strings
%        	- linked lists
%       		- search
%        		- insert
%        		- insert after
%        		- delete
%        		- delete bug-free
%        		- deleteThis
%        		- more issues
%       	- double linked lists
%        	- iterators
%        	
%        	_CHAPTER SIX_ Stacks and queues
%        	- properties
%        	- recall adts
%        	- application
%        		- tree traversal (xml)
%        		- depth first -> stacks
%        		- breadth first -> queues
%        		- space and time complexity
%        	- tower of hanoi
%        		- example
%        		- complexity
%        		- recursion
%        		
%        	_CHAPTER SEVEN_ sorting
%        	- assumptions and definitions
%        	- variations:
%        		- external vs internal
%        		- in-place vs add memory
%        		- pre-sorting
%        	- selection sort
%        	- insertion sort
%        	- bubble sort
%        	- lower bound
%        	- (full) decision tree
%        		- optimal set of comparison
%        		- shortest longest path
%        	- first summary?
%      	- merge sort
%      		- algorithm
%      		- illustration
%      		- example: merge two sorted lists
%      		- complexity, remarks
%      	- quick sort
%      		- main idea
%      		- algorithm
%      		- illustration
%      		- mean/median
%      		- pivot element k
%      		- QS in place
%      		- example
%      		- complexity
%      			- WC complexity
%      			- AC complexity
%      			- induction
%      			- conclusion
%      		- improving space complexity
%      		- further improvement
%      	- radix exchange sort 
%      		- knowledge on the nature of the values
 %     		- sorting complete binary strings
 %     		- illustration
%      		- complexity
%      	- bucket sort
%      		- example
%      		- complexity
%      		- space problem
%      	
%      	_CHAPTER EIGHT_ searching
%      	- searching in unsorted lists
%      	- searching in sorted lists
%      		- binary search
%      		- iterative binary search
%      		- fibonacci search
%      		- interpolation search
%      	- selecting in unsorted lists
%      		- choosing p
%      		- median-of-median
%      		
%      	_CHAPTER NINE_ self organizing lists
%      	- zipf distribution
%      	- fixed frequencies
%      	- dynamic frquencies, caching
%      	- organization strategies
%      		- MF move to front
%      		- T transpose
%      		- FC frequency count
%      	- average costs of strategies
%      	
%      	_CHAPTER TEN_ amortized analysis
%      	- example
%      	- approach:
%      		- accounting analysis
%      		- potential method
%      	- dynamic tables
%      	- SOL analysis
%      		- re-organization strategies
%      		- Notation
%      		- Theorem
%      		- goal and ideas
%      		- preliminaries
%      		- proof (short)
%      	- aggregation
%      	
%      	_CHAPTER ELEVEN_ priority queues
%      	- shortest path in a graph
%      	- exhaustive solution
%      	- dijkstra
%      	- PQ in ADTs
%      		- linked list
%      		- arrays
%      	- using heaps
%      		- analysis + illustration
%      		
%      	_CHAPTER TWELVE_ hashing
%      	- hash functions
%      		- definition & requirements
%      		- k mod a
%      		- prime
%      		- multiplicative method
%      		- two cakes a day? + analysis
%      	- collisions
%      		- analysis
%      		- overflow hashing
%      		- open hashing
%      		- dynamic hashing
%      	- external collision handling
%      		- seperate chaining
%      		- direct chaining
%      	- Bloom filter
%      		- false positive
%      		- average case
%      	
%      	_CHAPTER THERTEEN_ open hashing
%      	- definition usw
%      	- open vs external collision handling
%      	- searching, deletions
%      	- linear probing
%      	- quadratic hashing
%      	- double hashing
%      	- ordered hashing
%      	- Brent's Algorithm
%      	- dynamic hashing
%      	
%      	_CHAPTER FOURTEEN_ trees
%      	- decision trees
%      	- suffix-trees
%      	- graphs
%      		- trees as connected graphs
%      		- rooted trees
%      		- terminology
%      	- search trees
%      		- definition
%      		- searching
%      		- inserting
%      		- deleting
%      		- analysis (short)
%      	- avl trees
%      		- definition 
%      		- searching
%      		- inserting
%      		- deleting
%      		- analysis (short)
%      	
%      	_CHAPTER FIVTEEN_ optimal search trees
%      	- scenario
%      	- definition and request model
%      	- constructing an optimal search tree
%      	- towards divide and conquer
%      	- induction !?
%      	- implementation and analysis
%      	- search strings: tries
%      	
%      	_CHAPTER SIXTEEN_ graphs
%      	- coloring problem
%      	- seven bridges of königsberg
%      	- defintions! and types of graphs
%      	- data structures:
%      		- adjacency matrix
%      		- adjacency lists
%      	- transitive closure
%      	- graph traversal
%      		- breadth-first traversal
%      		- depth-first traversal 
%      		- where do we start problem
%      		- analysis
%      	- negative cycle paths
%      		- no dijkstra
%      		- first approach
%      		- naive solution
%      		- improvement
%      		- Warshall's Algorithm  tansitive closure
%      			- Correctness
%      			- Example
%      		- Floyd's Algorithm 	shortest path
%      			- Example
%      		- Strongly connected components
%      			- definition
%      			- graph traversal, method
%      			- Kosaraju‘s Algorithm
%      				- Correctness
%      				- Examples
%      				- Complexity
%      		
%      	_CHAPTER SEVENTEEN_ minimal spanning tree
%      	- defintion
%      	- Proof?
%      	- Prim‘s algorithm
%      	- Kruskal’s Algorithm
%      	- Boruvka‘s Algorithm
%      	
%      	_CHAPTER EIGHTTEEN_ knapsack problem
%      	- defintion
%      	- complexity
%      	- variations
%      	- idea
%      	- dynamic programming solution
%      	- analysis
%      	- approximate solution
      		
      	
      	

    %\backmatter
        \appendix
        \chapter{The first appendix}

Lorem ipsum
        %\cleardoublepage
%\thispagestyle{plain}
\chapter*{Acknowledgement}
lorem ipsum


        
        %\nocite{*}                
		% See the documentation or under
		% http://edoc.hu-berlin.de/e_autoren/latex/bedingung.php
		% for the list of the permitted styles.
        \bibliographystyle{apalike}
        \bibliography{bibliography}
        %\listoffigures
        %\listoftables    
        %% You can change this text, if needed.
\chapter*{Selbst�ndigkeitserkl�rung}
\selectlanguage{ngerman}
Ich erkl�re, dass ich die vorliegende Arbeit selbst�ndig und nur unter Verwendung der angegebenen Literatur und Hilfsmittel angefertigt habe.

\vspace{2\baselineskip}
\noindent Berlin, den \today\hfill\authorfirstname \authorsurname
\selectlanguage{english}
    % Use our template or write your own.
        %\include{chapters/CV}
\end{document}
